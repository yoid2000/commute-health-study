\section{Introduction}
\label{sec:introduction}

A goal is for researchers to be able to safely share with each other research data pertaining to individuals.

The data must of course be adequately anonymous and must produce correct scientific analyses.

Other features as well though:
\begin{itemize}
    \item Easy to generate the anonymous data (ideally fully automated with no anonymization expertise needed).
    \item Not necessary to individually approve each data release (ideally one blanket approval for the anonymization process).
    \item Easy to use the anonymized data for analytics (ideally, no difference between analysis using the original data and the anonymized data)
\end{itemize}

Synthetic data has been proposed as an attractive solution. A key advantage of synthetic data is that it is syntactically similar to the original data, and can therefore be directly used in a variety of data analysis tools.

This paper examines the suitability of several synthetic data methods for the purpose of data sharing. 

Methodology is to take an existing analysis of an original dataset, to apply that analysis to synthetic datasets, and determine whether the same scientific conclusions are reached.

For this purpose, we use the paper~\cite{jurak2021associations}.

Etc. etc.

\TODOpf{Finish first draft of intro}